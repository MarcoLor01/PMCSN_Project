\documentclass{article}
\usepackage{graphicx} % Required for inserting images
\usepackage{titling}  % Permette una maggiore personalizzazione del titolo
\usepackage{multicol} % Per scrivere su due colonne

% Pacchetto per la gestione del formato delle sezioni
\usepackage{titlesec}

% Imposta lo spazio sopra e sotto le sezioni
\titlespacing*{\section}{0pt}{0pt}{\baselineskip}

% Rimuove la numerazione dalle sezioni
\setcounter{secnumdepth}{0}

\title{Performance Modeling Of Computer Systems And Networks}
\author{
  Marco Lorenzini \\ 0353515
  \and
  Eugenio Di Gaetano \\ YYYYYYYY
}

\date{Università degli studi di Roma Tor Vergata} % Rimuove la data dal titolo

\begin{document}

\maketitle

\section{Abstract}
Il presente studio analizza la gestione di un Pronto Soccorso, focalizzandosi in particolare sul centro Policlinco Tor Vergata. Il Pronto Soccorso è un'unità cruciale all'interno di un ospedale, dedicata all'accoglienza e alla prima assistenza di pazienti in condizioni urgenti e non programmabili. Nello studio viene analizzato l'intero processo che ogni paziente che si rivolge ad un Pronto Soccorso deve attraversare, partendo dal triage iniziale fino alla dimissione o al ricovero ospedaliero.

\begin{multicols}{2}

\section{1. Introduzione}
Il sistema analizzato è il Policlinico Tor Vergata, il pronto soccorso deve gestire un flusso di pazienti derivanti da arrivo diretto in struttura o da arrivo tramite ambulanza. Il processo di triage e accoglienza rappresenta il primo contatto del paziente con il Pronto Soccorso. Questa fase determina la priorità del trattamento in base alla gravità delle condizioni del paziente e all'urgenza del caso, ai pazienti viene assegnato qui uno dei seguenti codici: \newline \newline 
Livello 1 (Rosso): Interruzione o compromissione di una o più funzioni vitali. \newline
Livello 2 (Arancione): Funzioni vitali a rischio, rischio evolutivo o dolore severo. \newline
Livello 3 (Azzurro): Condizione stabile con sofferenza. \newline
Livello 4 (Verde): Condizioni stabili, richiede prestazioni monospecialistiche. \newline
Livello 5 (Bianco): Problema non urgente o di minima rilevanza clinica. \newline \newline
Dopo il triage, il paziente è assegnato alla prima visita medica. Durante questa visita, il medico stabilisce le condizioni del paziente e pianifica gli esami diagnostici e i trattamenti necessari. Questa fase è fondamentale per garantire un'assistenza mirata e tempestiva, adeguata alle necessità specifiche del paziente.
Al termine della prima visita, il medico valuta se i controlli e le cure effettuati sono sufficienti per permettere al paziente di lasciare il Pronto Soccorso con le adeguate indicazioni per il follow-up. In caso contrario, il paziente può essere ricoverato in ospedale per ricevere cure specialistiche continue e monitoraggio delle condizioni di salute. \newline


\section{2. Problematiche del sistema}


\columnbreak


\end{multicols}

\end{document}
